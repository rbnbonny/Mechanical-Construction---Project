\chapter{Mode d'emploi}
Le fonctionnement de la machine est assuré par une seule personne, grâce à la manivelle. Avant de commencer le tri, l'utilisateur doit verser avec précaution les 300 billes dans le réservoir pour billes mixtes, situé sur le dessus de la machine. Il peut alors les répartir avec la main sur le fond du réservoir, s'il constate un empilement important avant le distributeur.

Pour débuter le tri des billes, l'utilisateur tourne avec sa main droite la manivelle dans le sens des aiguilles d'une montre. Il peut poser sa main gauche sur le réservoir mixte sans danger, ce qui stabilise d'avantage la machine. Il ne faut cependant pas mettre les doigts dans la partie découverte du distributeur lorsque l'on tourne la manivelle, pour des raisons de sécurité. La vitesse de rotation de la manivelle doit être d'environ un tour par seconde, pour assurer un tri optimal des billes. Le temps nécessaire pour trier 300 billes est d'environ ().

A la fin du processus, l'utilisateur peut récupérer les billes séparées en fonction de leur taille dans trois réservoirs distincts situés dans le fond de la machine. Depuis le côté du distributeur, le premier réservoir contient les billes de 5 mm, le deuxième celles de 6 mm, et le dernier les billes de 7 mm. Ces réservoirs peuvent facilement être sortis et remis dans leur emplacement.