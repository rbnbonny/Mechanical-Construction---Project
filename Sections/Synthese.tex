\chapter{Synthèse}
Pour notre projet de construction mécanique, nous devions construire une machine capable de trier des billes de roulements, selon toutes les exigences du cahier des charges. Étant quatre étudiants de première année, nous avons essayé d'imaginer des concepts capables de répondre au plus près à ces exigences. Plusieurs idées ont émergées, chacune avec ses avantages et ses inconvénients. Pour chaque partie de la machine, nous avons dû trouver des solution adaptée à des problèmes différents, avec un objectif principal: prévenir tout risque de coincement.

Une fois la machine et ses concepts fixés, nous nous sommes partagés le travail, pour gagner en productivité. Régulièrement, nous nous sommes retrouvés pour une mise en commun afin de surveiller l'avancée du travail, de poser nos questions et d'essayer d'y répondre en groupe. De plus, il fallait aussi régulièrement se mettre d'accord sur la manière de rédiger le rapport et d'illustrer notre propos, pour assurer l'unité du travail. 

Il nous a fallu gérer le dessin des pièces sur \emph{CATIA}, l'assemblage final, la mise en plan de chacune des pièces, et la rédaction du rapport. Chacune de ces étapes a son importance, et nous avons fait de notre mieux pour rendre un travail de qualité.

\section{Vérification des spécifications} Est-ce que la machine finalement satisfait les donnés du cahier des charges et des spécifications définis au début?

\begin{table}[htbp]
    \centering
    \begin{tabular}{|c|c|c|c|}
        \hline
         & \textbf{Cahier des charges} & \textbf{Spécifications} & \textbf{Résultats}\\
        \hline
        Dimensions & facile à transporter & < A4 & longueur \SI{252.6}{\milli\metre} \\
        & & & largeur \SI{189.71}{\milli\metre} \\
        & & & hauteur \SI{133.7}{\milli\metre} \\
        \hline
        Masse & facile à transporter & \SI{< 4}{\kilo\gram} & \SI{2.565}{\kilo\gram} \\
        \hline
        Temps & le plus cout possible & \SI{< 30}{\second} & \SI{0}{\second} \\
        \hline
        Matériau & matériaux courants & aliage Al & EN AC-AlSi7Mg0.3 T6 \\
        & & & X 10 Cr Ni S 18 10 \\
        \hline
        Source & manivelle & manivelle & manivelle avec \\
        d'énergie & & & rayon \SI{32}{\milli\metre}\\
        \hline
    \end{tabular}
    \caption{Les donnés du cahier des charges, les spécifications résultantes et les résultats calculés de la machine}
    %\label{tab:specifications}
\end{table}

Les objectifs de la dimension, de la masse, du temps de tri, du matériau et de la source d'énergie sont faciles a vérifier et cadrent avec le cahier des charges. Malgré que l'objectif d'empêcher le coincement des billes est difficile a vérifier avec une grande certitude, le concept de la machine le prend en compte et nos calculs montrent que les billes se devraient théoriquement pas coincer. 