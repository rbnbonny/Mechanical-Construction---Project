\chapter{Synthèse}
Pour notre projet de mécanique, nous devions construire une machine capable de trier des billes de roulements, selon toutes les exigences du cahier des charges. Etant quatre étudiants de première année, nous essayé d'imaginer des concepts capables de répondre au plus près des exigences. Plusieurs idées ont émergées, chacune avec ses avantages et ses inconvénients. Pour chaque partie de la machine, nous avons dû trouver des solution adaptée à des problèmes différents, avec un objectif principal: prévenir tout risque de coincement.

Une fois la machine et ses concepts fixés, nous nous sommes partagés le travail, pour gagner en productivité. Régulièrement, nous nous sommes retrouvés pour une mise en commun afin de surveiller l'avancée du travail, de poser nos questions et essayer d'y répondre en groupe. De plus, il fallait aussi régulièrement se mettre d'accord sur la manière de rédiger le rapport, d'illustrer notre propos, pour assurer l'unité du travail. 

Il nous a fallu gérer le dessin des pièces sur \emph{CATIA}, l'assemblage final, la mise en plan de chacune des pièces, et la rédaction du rapport. Chacune de ces étapes a son importance, et nous avons fait de notre mieux pour rendre un travail de qualité.