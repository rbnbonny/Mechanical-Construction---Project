\chapter{Justifications des choix}

\section{Réservoir pour les billes mixtes}

\subsection{Dimensionnement}
Les parois du réservoir initial mesurent \SI{15}{\milli\metre} de hauteur: associées au fond en feutre, elles permettent d'éviter que des billes ne soient perdues si elles sont versées rapidement, tout en économisant de la matière par rapport à des parois sur-dimensionnées. 
Ensuite, la pente de \ang{3} du fond du réservoir assure la descente des billes vers le distributeur, mais avec une vitesse modérée. De plus, les interactions des billes entre elles renforcent ce phénomène de ralentissement.
Finalement, nous avons calculé la surface du réservoir comme suit:
\[S = L \cdot l = \SI{99}{\milli\metre} \cdot \SI{165}{\milli\metre} = \SI{16335}{\milli\metre\squared}\]
Où $L$ est la longueur du fond du réservoir, et $l$ la largeur.
Ainsi, dans le pire des cas, soit \num{300} billes de \SI{7}{\milli\metre} de diamètre, ces dernières peuvent se répartir sur l'ensemble du fond du réservoir sans se superposer, si elles sont versées avec soin. En effet, la surface de répartition est donnée par:
 Nous avons calculé le volume des réservoirs comme suit:
\[S\textsubscript{max} = (\SI{7}{\milli\metre})^{2} \cdot \num{300} \cdot d \approx {\num{7}}^{3} \cdot 300 \cdot 0.9 = \SI{13230}{\milli\metre\squared}\]
Où \(d = \frac{ \pi }{2\sqrt{3}} \approx \num{0.75}\) est la densité surfacique de l'arrangement compact de billes.
Cela permet déjà de réduire le risque de problèmes au niveau du distributeur, même si d'autres précautions ont été prises plus loin pour exclure tout coincement.

\subsection{Assemblage}
Pour assembler le fond du réservoir avec les parois, nous avons choisi d'utiliser des goupilles, car des vis aurait exigé un perçage chanfreiné des parois du réservoir pour fixer une vis à tête conique, mais l'épaisseur de \SI{6}{\milli\metre} ne permet pas: cela aurait affaibli le montage. A l'opposé, les goupilles peuvent être fixées grâce à un simple perçage, ce qui renforce l'assemblage.

\section{Acheminement}
\subsection{Choix final}
Après l'évaluation de toutes les options notre choix fut le suivant:
\begin{itemize}
	\item \num{12} cylindres perforés de diamètre \SI{60}{\milli\metre} avec \num{10} trous arrondis de profondeur \SI{5.586}{\milli\metre} montés sur une barre
%à discuter
	\item une barre de diamètre \SI{8}{\milli\metre} avec un collet (diamètre \SI{6}{\milli\metre}) aux deux extrémités pour la fixer dans deux roulements à billes. À l'extrémité droite se trouve un taraudage (M4) pour fixer la pièce tenant la manivelle.
	\item une grande fixation quasi-symétrique des deux cotés. Elle contiennent un évidement pour serrer les roulements à billes, des extrusions avec des taraudages pour fixer le cache et deux extrusions pour fixer le réservoir mixte et les rails. La seule différence entre la fixation du côté gauche et celui du côté droite est qu'à droite au centre de l'évidement se trouve un alésage passant à travers pour pouvoir visser la fixation de la manivelle à la barre 
	\item une pièce pour tenir la manivelle qui consiste d'un cylindre avec un carré d'entraînement d'un côté et un filetage M4 de l'autre
	\item une manivelle préfabriquée (Crank Handholes Straight Type QB-N) avec $K = 8$ et du coup une longeur totale (L) de \SI{49}{\milli\metre} voir un radius de rotation de \SI{32}{\milli\metre}.
	\item un couvercle empêchant que les billes tombent dans les derniers \ang{30} de la rotation.
\end{itemize}


\subsubsection{Vitesse de tri}
La surface de contact entre les cylindres perforés et le réservoir est de \SI{96}{\milli\metre}. Le diamètre moyen d'une bille est \SI{6}{\milli\metre}. Sur la surface de contact se trouveront alors

\[\frac{\num{96}}{\num{6}} = \num{16}\]
billes. À la surface de contact, il y a \num{8} trous à la fois à disposition qui seront donc tous remplis. Ainsi, lorsque la manivelle fait un tour entier, on pourra potentiellement acheminer 
\[(\text{nombre de trous par cylindre}) \cdot (\text{nombre de cylindres}) = \num{10} \cdot \num{12} = \num{120}\]
billes.
On considère que jusqu'à ce qu'il ne reste que \num{50} billes dans le réservoir, la surface de contact sera remplie entièrement (pour avoir encore plus de certitude calculera avec une marge de sécurité de \num{0.9}).
Alors si on commence dans une position de départ où les billes peuvent directement entrer et qu'on achemine \num{120} billes par tour on va

% SIUNITX fixed until here (Robin)

\[\frac{\theta}{\omega} = \frac{\frac{(250 \cdot 2)\pi}{120 \cdot 0.9}}{\frac{2\pi}{s}} = \frac{250}{108} \ s = 2.315 \ s \]
acheminer 250 billes en \SI{2.315}{\s} avec une vitesse de rotation d'un tour par seconde.
Les \num{50} dernières billes vont s'acheminer un peu plus lentement vu que la probabilité qu'ils tombent directement sur un trou diminue. Au lieu de faire un calcul exponentiel très compliqué on va partir de l'approximation qu'ils soient entraînes avec une certitude de \SI{50}{\percent}. Ce qui fait que
\[\frac{\theta}{\omega} = \frac{\frac{(50 \cdot 2)\pi}{120 \cdot 0.5}}{\frac{2 \pi}{s}} = \frac{50}{60} \ s = 0.833 \ s\]
0.833 secondes supplémentaires suffiront.
En un total de \SI{3.148}{\sec} toutes les billes auront alors passé le mécanisme d'acheminement.

\subsection{Acheminement}
La première question qui se pose, c'est si les billes de diamètre 5 vont rester dans leurs cases pendant les premiers \ang{150} de leur tour (cf. figure en dessous). Car si la condition centripète n'est pas remplie constamment, la force centrifugale (dans le repère tournant) dirigerait les petits billes vers les arêtes des dents. Donc il y aura un contact qui créerait une perte par résistance au roulement (vu que c'est des billes de roulement on va considérer qu'ils vont rouler sans glissement). Si nous considérons la figure ci-dessous, l'accélération centripète est donnée par:
\[ma_{c} = mg \cdot \sin\theta\]
Pour une vitesse de rotation de \(\frac{2 \pi}{s}\) la condition centripète reste remplie pour tout angle supérieur à %\ang{?}
\[\cancel{m} \cdot g \ sin\theta = \cancel{m} \frac{v^{2}}{R + r} = \cancel{m} \cdot \omega^{2} \ (R +r)\]
\[sin \theta = \frac{\omega^{2} \ (R + r)}{g} = \frac{\frac{2 \pi}{s^{2}}\ (24.414mm + 2.5mm)}{10\frac{m}{s^{2}}} = \frac{\frac{4 \pi^{2}}{s^{2}} \ (26.914mm)}{10\frac{m}{s^{2}}} = 0.0388\]
\[\theta = \arcsin(0.0338) = \]
Il reste à vérifier deux choses. Est-ce que le centre de masse de la bille restera dans la zone verte (voir figure en dessous) entre $\theta$ = voir en haut et $\theta$ = 0.
%figure manque
\subsubsection{Partie Pink 2}
Vu que dans cette partie de la rotation la condition centripète n'est plus satisfaite, la bille va commencer à rouler hors de son logement. Pour faciliter les calculs on va considérer que la bille soit un point de masse qui se déplace sans friction en direction du couvercle pendant $\theta = $%voir en haut
et $\theta = 0$. 
Cette approximation assez forte est tout à fait valable. Car si je calcule l'accélération de la bille sans prendre la friction, la résistance au roulement et l'énergie convertie en énergie de rotation en compte je majore celle-ci. C'est-à-dire que si la bille ne sort pas de la zone verte dans le calcul aproximé, elle ne le ferait sûrement pas en réalité.\\
On a alors l'équation pour la deuxième loi de Newton suivante (dans le repère tournant):
\[ma = m \frac{\omega^{2}}{(R + r)}\]
\[a = \omega \ (R + r) = \frac{4 \pi^{2}}{s^{2}} \ (26.914mm) = \]
%à calculer et si nécessaire inclure friction
Le déplacement $\Delta\theta$ entre $\theta =$ voir en haut et $\theta = 0$ va prendre
\[2 \pi \estimates 1 s\]
\[\rightarrow t = \frac{x \ s}{2 \pi} = \]
et donc la bille se déplacera de 
\[s = \frac{1}{2} \ a \ t^{2} = ? \]
Ceci est inférieur à la distance 
\[d = b - 2r = 8.086mm - 2 \times 2.5mm = 3.086mm\]
où b est la distance entre le fond et le couvercle et on en soustrait 2r cas le centre de masse se trouve à distance r du fond et la bille n'a justement pas le droit de se trouver qu'à distance r du couvercle).

\subsubsection{Partie Pink 3}
Dans les derniers \ang{45} la condition centripétale ne sera plus remplie. Ceci va créer un contact avec l'intérieur de la perforation et avec le couvercle. Mais vu que même pour les billes de radius \SI{5}{\milli\metre} le centre de masse ne pourra pas dépasser le rayon rouge (cf. image ci-dessous) cela ne bloquera pas le mécanisme.
%insérer image
Par contre ceci créera une perte d'énergie qui sera traité dans le paragraphe suivant.

\subsection{Consommation d'énergie et sollicitation}
Pour garantir au consommateur une utilisation facile et un fonctionnement durable il reste à vérifier si notre mécanisme se laisse opérer avec un effort acceptable et s'il y a pas des frictions fortes qui vont créer des usures néfastes trop vite. L'acheminement fut conçu en minimisant les surfaces de contact avec les éléments statiques afin de minimiser la friction. En considérant le dessin ci-dessous on s'aperçoit que tout les moments de force en jeu se résument dans l'équation suivante:
%dessin manque
%formule manque
où alfa (?) %formule 
et où on considère le pire des cas ou on a toute les \(10 \cdot 12\) cases remplis où pendant les derniers \ang{45} toutes les billes ont du contact avec le couvercle et l'arête de la dent et ainsi une résistance au roulement et de la friction. La résistance au roulement et le frottement dans le roulement à bille est négligée.
En résolvant ceci selon $F_{main}$ on obtient
%formule fmain =
Le travail fourni pas l'utilisateur est alors 
% fmain * 2 pi
L'énergie investi par cycle suit ainsi du calcul
%formule
\subsection{Déblocage}
Finalement il reste à évaluer s'il y a un empilement possible dans les empoches des cylindres. Un tel empilement aurait comme conséquence que les billes soient très proche sur les rails et se freinent les uns les autres où même que sa bloque l'acheminement.

\subsubsection{Calculs}
L'idée initiale fut de mettre le couvercle à une distance de 2.5 mm des cylindres perforés pour que s'il y a un empilement de deux billes de 5mm (ce qui représente le seul cas problématique) la bille de dessus soit heurté très proche de son centre de masse et ainsi repoussée.
Mais considérons le dessin ci-dessous \ang{35} de l'horizontale, c'est-à-dire \ang{5} de rotation après que la bille soit soulevée)
%Dessin manque
Si on impose l'équilibre sur la bille extérieur
%formules
on s'aperçoit que \textit{a} devrait être inférieur à (?) pour que la bille reste en position. Par contre en pratique \textit{a} vaut (?) ce qui signifie que les billes ne s'empileront pas. Et vu que l'acheminement se fait en 3.148 secondes tout empilement devant à la surface de contact sera négligeable.

\section{Séparation et tri}
Pour la séparation et le tri des billes nous avons choisi la parties avec les rails de tri en avant et les rainures de séparation en arrière, tout combiné dans une seule pièce.

\subsection{Rainures}
Chaque des six rainures unit les billes de deux roues et amène-les sur un rails pour le tri. Pour leur forme nous avons choisi un composé de deux triangles, chaque avec \SI{5}{\milli\metre} de haut et \SI{8}{\milli\metre} de large, égale à la largeur d'une roue. Ces triangles se trouvent arrangés de manière symétrique de chaque côté de la rainure centrale. Les rainures ne sont pas fraisés perpendiculaires à leur surface normale, mais parallèles à la pente de la pièce. L'angle d'inclinaison des rainures, dont nous avons besoin pour calculer les points de contact est donné par: \[\alpha_{\text{rainure}} = \arctan\left(\frac{\SI{5}{\milli\metre}}{\SI{8}{\milli\metre}}\right) \approx \num{0.5586} \approx \ang{32}\]

\subsection{Pente}
La pente est donnée par une différence de \SI{2}{\milli\metre} sur la longueur totale de \SI{169}{\milli\metre} de la pièce, ce qui mène à un angle d'élévation de:
\[\alpha_{\text{pente}} = \arctan\left(\frac{\SI{2}{\milli\metre}}{\SI{169}{\milli\metre}}\right) \approx \num{0.0118} \approx \sisetup{add-arc-degree-zero} \ang{;40;}\]

\subsection{Vitesses angulaires}
Sur les rainures, la distance entre l'axe de rotation des billes et leur points de contact est donnée par la formule suivante, $d$ étant le diamètre de la bille:

\[s_{\text{rainure}}(d) = \cos(\alpha_{\text{rainure}}) \cdot \frac{d}{2} = \frac{1}{\sqrt{\left(\frac{\SI{2}{\milli\metre}}{\SI{169}{\milli\metre}}\right)^2+1}} \cdot \frac{d}{2}\]

\[s_{\text{rainure}}(\SI{5}{\milli\metre}) = \cos(0.5586) \cdot \frac{\SI{5}{\milli\metre}}{2} \approx \SI{2.12}{\milli\metre}\]

\[s_{\text{rainure}}(\SI{6}{\milli\metre}) = \cos(0.5586) \cdot \frac{\SI{6}{\milli\metre}}{2} \approx \SI{2.54}{\milli\metre}\]

\[s_{\text{rainure}}(\SI{7}{\milli\metre}) = \cos(0.5586) \cdot \frac{\SI{7}{\milli\metre}}{2} \approx \SI{2.97}{\milli\metre}\]

Sur les rails, la distance entre l'axe de rotation des billes et leur points de contact est donnée par la formule suivante, où $b$ est la distance entre les deux rails et $d$ le diamètre de la bille:
%\[s_{\text{rail}}(d) = 1 - \cos^2\left(\arctan\left(\frac{d}{2}\right)\right)\]

\[s_{\text{rail}}(d,b) = \sin\left(\arccos\left(\frac{b/2}{d/2}\right)\right) \cdot \frac{d}{2}\]

\[s_{\text{rail}}(\SI{6}{\milli\metre},\SI{5.2}{\milli\metre}) = \sin\left(\arccos\left(\frac{\SI{5.2}{\milli\metre}/2}{\SI{6}{\milli\metre}/2}\right)\right) \cdot \frac{{\SI{6}{\milli\metre}}}{2} \approx \SI{1.50}{\milli\metre}\]

\[s_{\text{rail}}(\SI{7}{\milli\metre},\SI{5.2}{\milli\metre}) = \sin\left(\arccos\left(\frac{\SI{5.2}{\milli\metre}/2}{\SI{6}{\milli\metre}/2}\right)\right) \cdot \frac{{\SI{7}{\milli\metre}}}{2} \approx \SI{2.34}{\milli\metre}\]

\[s_{\text{rail}}(\SI{7}{\milli\metre},\SI{6.2}{\milli\metre}) = \sin\left(\arccos\left(\frac{\SI{6.2}{\milli\metre}/2}{\SI{7}{\milli\metre}/2}\right)\right) \cdot \frac{{\SI{7}{\milli\metre}}}{2} \approx \SI{1.62}{\milli\metre}\]


\section{Réservoir pour les billes triées}
La version finale du réservoir pour les billes triées était le troisième brouillon. Au contraire du premier brouillon, la version finale nous permet d'enlever les containers avec les billes triées séparément, et cela sans que la construction des pièces soit difficile, ce qui a exclu le deuxième brouillon. Voici ces deux raisons principales en plus de détail:

\subsection{Containers flexibles}
Quelle que soit l'utilisation des billes triées, la flexibilité des containers, qui permettent d'enlever toutes les billes triées de la même taille en même temps, est bien utile lors de l'utilisation future de la machine. Son efficacité en service est donc bien meilleure que si l'on devait enlever toutes les billes à la main d'un container fixe. Le positionnement exact des containers est garanti par une plaque %{insérer lien dessin sol}
qui entoure les pieds de chacun des containers. Le jeu pour enlever les containers est d'un millimètre en haut ainsi qu'en bas. C'est assez grand pour qu'ils ne se coincent pas, et assez petit pour que la distance entre le haut du container et la prochaine pièce soit de 4 mm, ce qui garantit que même les billes les plus petites trouvent leur chemin dans le bon réservoir, une fois tombées à travers les rails.

\subsection{Usinage}
Nous avons décidé d'augmenter le nombre de pièces afin de simplifier leur usinage. Parmi les pièces utilisées, il n'y a aucune pièce qui ne pouvait pas être usiné par des processus courants, certaines même sans assistance par ordinateur. La décomposition en pièces faciles à usiner est un avantage pour la production, mais pas pour l'assemblage de la machine. Si l'on voulait lancer la machine en production en grande série, il faudrait reconsidérer ce choix. En grand nombre, même des pièces plus complexes sont moins chères car les coûts de la configuration de l'usinage peuvent être répartis sur plusieurs pièces et forment donc une plus petite partie du prix de la pièce. Nous avons estimé que pour notre projet, le surcroît des dépenses se compense facilement par l'économie réalisée pendant l'usinage.

En outre, nous avons fait intervenir des parois qu'il est possible d'acheter bon-marché. En découpant une paroi, nous obtenons une pièce qui s'utilise facilement comme partie d'un container. Comme déjà évoqué, nous avons rejeté l'idée d'ajuster la taille de chaque réservoir à la taille des billes correspondantes, pour augmenter la simplicité dans la production. 

\subsection{Dimensionnement}
Le dimensionnement des cubes se déroulait en prise en compte de la taille des billes. Dans le pire des cas, les 300 billes sont de la même taille : les réservoirs doivent alors être capables de contenir 300 billes de la taille respective. Pour simplifier la construction, les cubes ont tous la même taille sans respecter la taille des billes.
Nous avons calculé le volume des réservoirs comme suit:
\begin{equation}
    V = 7^3\ 300\ d \approx 7^3\ 300\ 0.75 = 77'175\ mm^3
    \label{eq:volumereservoirfin}
\end{equation}

Où \(d = \frac{ \pi }{3\sqrt{2}} \approx 0.75\) est la densité volumique de l'empilement compact de billes.
La taille du rayon interne de la paroi se calcule comme suit:

\begin{equation}r\textsubscript{int}\geq\sqrt{\frac{2\ V}{\pi\ l}}=\sqrt{\frac{2\ 77'175\ mm^3}{\pi\ 90\ mm}} \approx 23.4\ mm \label{eq:rayonreservoirfin}\end{equation}

Avec \(V=77'175\ mm^3\) calculé par la formule \ref{eq:volumereservoirfin} et \(l=90\ mm\) la longueur interne adaptée à la largeur des rails. En ajoutant un petit facteur de sécurité si les billes ne s'empilent pas de la manière la plus compacte possible, et en fixant l'épaisseur de la paroi de \(2\ mm \), nous avons obtenu un rayon externe de la paroi de \(r\textsubscript{ext}=27\ mm\). Pour que les billes ne fassent pas trop de bruit lors du choc au réservoir, nous avons prévu un revêtement en feutre d'une épaisseur d'un millimètre. Le rayon interne mesure alors \(24\ mm \), ce qui respecte toujours la limite qu'on a calculé par la formule \ref{eq:rayonreservoirfin}.
